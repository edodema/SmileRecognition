\section{Task and motivations}
The goal of this project is developing a facial emotion recognition system, facial emotion recognition ~\cite{blog:facial_recognition} is the process of analyzing someone's human state from facial expressions.
This is done automatically by the brain but software has been and is developed such that machine could recognize human emotions too. 
The technology is improving all the time and some are confident it will be able to read emotions as our brains do .
Emotion AI has a wide range of uses from scanning for signs of terrorism ~\cite{blog:terrorism} to emotion judgement for commercial purposes as \textit{Disney} did or at least wanted to do for the release of \textit{Toy Story 5} ~\cite{blog:disney}.
All that glitters is not gold and Emotion AI could be a threat to freedom and equality ~\cite{blog:china1}, in China \textit{Hanwang Technology} deployed a system that tracks students' behaviour during lessons analyzing if they're keeping attention or their action (e.g. "Answering question") ~\cite{blog:china3}. 
This software seems to be unefficient and based on pseudoscientific assumptions, these allegations didn't stop \textit{Amazon}, \textit{Microsoft} and \textit{Google} from offering emotion recognition to their customers even though the first two companies claim that their product can't determine a person's internal emotional state from only facial expressions ~\cite{blog:china2}. 

The ethical dilemma of these technologies can be reduced to the old saying \textit{quis custodiet ipsos custodes?}. 