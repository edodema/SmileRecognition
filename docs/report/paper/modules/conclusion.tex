\section{Conclusions}
Starting from the fact that I did not expect too high performances a $76\%$ AUC is not too bad, unfortunately I did not find any notebook or any other data that uses the same dataset as mine to compare my results with.
I expected the polynomial kernel to be better than the linear one and I was surprised of how a degree of 3 makes the model overfit.

Last the online demo seems to work better under precise conditions: is reccomended to make the face visible for face detection and glasses can compromise the expression identification. 
Brightness is crucial, not only for the emotion recognition phase but if the background features clutter for the face detection too.
After some experiments I noticed how the ideal distance is along 22 centimeters from the webcam, this could be due to two differents reasons: if the distance is shorter there can be false positive face detections, while if it is shorter or higher then the emotion recognition tends to fail and stick to an emotion.
Another interesting point is that three-quarter poses tend to be recognized as positive emotions even when they're not, this could be probably due to a selection bias in the dataset.

\subsection{Future works}
It could be possible to implement a classifier that predicts both valence and arousal, it could be done with a SVM multiclass classifier that uses softmax regression or with two binary SVMs.
Another possible improvement could be implementing the emotion detection with a nerual network and see how results compare.
