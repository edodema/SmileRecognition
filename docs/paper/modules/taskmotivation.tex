\section{Task and motivations}
Facial emotion recognition ~\cite{blog:facial_recognition} is the process of analyzing someone's human state from facial expressions, the human brain does it automatically but developing software able to recognize human emotions is a growing topic in research, the technology is constantly improving and some people are confident machines will be able to read emotions as our brains do .
Emotion AI has a wide range of use cases from scanning for signs of terrorism ~\cite{blog:terrorism} to emotion judgement for commercial purposes as \textit{Disney} did, or at least wanted to do, for the release of \textit{Toy Story 5} ~\cite{blog:disney}.

But all that glitters is not gold and these technologies could be a threat to freedom and equality ~\cite{blog:china1}, in China \textit{Hanwang Technology} deployed a system that tracks students' behaviour during lessons and analyzes their actions to infer if they're keeping attention or not ~\cite{blog:china3}.
This software seems to be unefficient and based on pseudoscientific assumptions, however these allegations didn't stop \textit{Amazon}, \textit{Microsoft} and \textit{Google} from offering emotion recognition to their customers. 
Even though the first two companies claim that their product can't determine a person's internal emotional state from only facial expressions ~\cite{blog:china2}. 

Other than the ethical dilemma that arises from these technologies is important to note how these are not real emotion recognition systems.
In the same way this project will not recognize real emotions but rather expressions, to be precise it aims to differentiate positive expressions from negative ones.
With this abuse of terminology let's proceed describing the strategy adopted against that task.